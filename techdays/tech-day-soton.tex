\documentclass[12pt,a4paper]{scrartcl}
\usepackage{url, booktabs,pdflscape}
\title{Student Robotics Tech Day Risk Assessment Form}

\begin{document}
\maketitle

\begin{description}
\item[Activity being assessed:] Student Robotics Tech Day 2014 (2014-12-06, 2015-02-21, 2015-03-21)
\item[Location:] Rooms 1007 and 1009, Building 25, Highfield Campus, University of Southampton (\url{http://data.southampton.ac.uk/building/25.html})
\item[Who is exposed to the hazard:] Competitors, Teachers, Mentors
\end{description}

\begin{description}
\item[Assessor's name:]
\item[Assessor's job title:]
\item[Assessor's signature:]
\item[Date of assessment:]
\end{description}
\clearpage

\newcommand{\risk}[3]{
 #1 & #2 & #3 \\
}

\begin{landscape}
\section{Risks}
The following risks have been considered for the student robotics competition.  Further description of the meaning of risk ratings (presented in this section as $L \times S$) can be found in the next section.

A safety briefing will be given on all mornings, covering the points below.

\bigskip
\begin{tabular*}{\linewidth}[c]{p{14em}p{30em}c}
\toprule
\textbf{Hazard} & \textbf{Control Measures} & \textbf{Risk Rating} \\
\midrule

\risk{Injury while using power or manual tools}
{Teachers to supervise all use of tools given that teams bring their own. These tools are used at the team's own risk. Student Robotics will not provide tools at any Tech Day.}
{3}

\risk{Electrocution by contact between water, electrical output and human}
{Water and electrical outputs kept strictly apart. Food and Drink is not allowed in the pit areas (i.e. places where teams work on their robots), or around the mini-arena.}
{3}

\risk{Risk of Fire}
{No naked flames are allowed to be used intentionally. If a fire breaks out accidentally, iSolutions regulations will be followed as described here: \url{http://www.southampton.ac.uk/isolutions/essentials/learnandteach/cls/fire.html}}
{2}

\risk{Interaction with robots: electric shock, minor injury.}
{Competitors are only allowed into the mini-arena with a mentor present, and may only work on robots in their pits. Robots may only be tested under supervision and if robot safety rules are met (see rulebook section 2). Electronics provided by Student Robotics are housed in a plastic casing, and wiring will be inspected by a member of Student Robotics before competitors are allowed to work on their robots.  Rulebook: \url{https://www.studentrobotics.org/resources/2015/rulebook.pdf}}
{1}

\risk{Misuse of batteries}
{See rules 2.16, 2.17. Batteries must only be charged as described here:
\url{https://www.studentrobotics.org/docs/kit/batteries}}
{2}
\bottomrule
\end{tabular*}
\end{landscape}

\input{assessment-guidance}

\clearpage
\appendix
\section{Fire Safety}
\textit{From iSolutions Regulations -- \url{http://www.southampton.ac.uk/isolutions/essentials/learnandteach/cls/fire.html}}

All Common Learning Spaces have a Fire Evacuation Route Poster located usually near the exit of in a glassed wall display cabinet alongside the other Common Learning Spaces signage.

\subsection{If you discover a fire}
\begin{enumerate}
\item Activate the alarm at any fire alarm call point by breaking the glass.
\item Evacuate the building by the most direct route.
\item Report to the assembly area
\end{enumerate}

\subsection{If you hear the alarm}
\begin{enumerate}
\item  Switch off any electrical equipment that you have been using, if safe to do so.
\item Close the door of the room when leaving.
\item Evacuate the building by the most direct route, and report to the assembly point.
\end{enumerate}

\end{document}

