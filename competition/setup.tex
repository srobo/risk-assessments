\documentclass[12pt,a4paper]{scrartcl}
\usepackage{url, pdflscape}
\usepackage[final]{pdfpages}
\usepackage[colorlinks]{hyperref}
\usepackage{longtable}
\title{Student Robotics Risk Assessment Form}

\begin{document}
\maketitle

\begin{description}
\item[Activity being assessed:] Student Robotics Competition (23 - 24 April, 2022)
\item[Location:] Southampton University Students Union (SUSU)
\item[Persons at risk:] SR Volunteers, SUSU Staff
\end{description}

\begin{description}
\item[Assessor's name:] Thomas Scarsbrook
\item[Responsible Persons:] Health and Safety Lead
\item[Date of assessment:] 02/04/2022
\end{description}
\clearpage

\newcommand{\risk}[4]{
 #1 & #2 & #3 & #4 \\
}

\begin{landscape}
\section{Risks}
The following risks have been considered for the setup of the Student Robotics Competition.
This assessment is to be supplimented by the SUSU risk assessment.
Further description of the meaning of risk ratings (presented in this section as
$L \times S$) can be found in the final section.

\centering
\begin{longtable}{|p{17em}|p{8cm}|p{4cm}|p{4em}|}
\hline
\textbf{Hazard} & \textbf{Control Measures} & \textbf{Responsible Person} & \textbf{Risk Rating} \\
\hline
\endhead

\endfoot

\risk{Injury while using manual or power tools}
{
Tools should only be used in appropriate circumstances and in the manner they are designed to be used.
All use should be supervised by a responsible adult.
First aid provision available to manage any incidents - provided by the means of SUSU staff and appropriately trained SR Volunteers.}
{Health and Safety Lead}
{4}
\hline

\risk{Injury due to persons or objects falling from height}
{
Work at height only conducted by suitably prepared individuals.
Personnel clipped on where appropriate.
Tools on lanyards where appropriate.
Head protection to be worn.
Area of worked barriered off.
}
{Health and Safety Lead}
{3}
\hline

\risk{Accidents due to fatigue from working long hours}
{
Individuals suspected of excessive tiredness restricted from activities that may be consequently dangerous.
Opportunity and space for breaks readily available.
}
{Health and Safety Lead}
{3}
\hline

\risk{Injury from improper manual handling}
{
Individuals involved in manual handling trained and briefed.
Appropriate protective equipment provided, if applicable.
}
{Health and Safety Lead}
{3}
\hline

\risk{Trip Hazard from trailing extension leads}
{Extension leads taped down and inspected regularly, kept away from walkways
where reasonably practicable. Blueshirts and Team Leaders to enforce teams
keeping within their areas and that areas are kept tidy}
{Health and Safety Lead, Team Leaders}
{1}
\hline

\risk{Falling on stairs}
{
Carrying of large or heavy objects on the stairs to be kept to a minimum. Running on the stairs is not permitted.
}
{Health and Safety Lead}
{2}
\hline

\risk{Injury due to objects falling from arena / arena components coming loose}
{
Arena to be constructed and tested as per Method Statement, and will
be subject to inspection by SR Volunteers throughout the event, with
interventions for repair if deemed necessary.
}
{Health and Safety Lead}
{3}
\hline

\risk{Hearing damage from excessive noise levels}
{Noise levels carefully monitored during event.}
{Health and Safety Lead}
{2}
\hline

\risk{Reaction to theatrical effects utilised, such as lighting effects}
{Theatrical effects will be limited to testing functionality.
Anyone present that may be affected by the test will be notified before the test to give them chance to ensure they are not at risk.
Flashing lights kept to a minimum. Any flashing to be at no greater rate than 4 flashes per second.}
{Health and Safety Lead}
{4}
\hline

\risk{Accidents due to being under the influence of alcohol or drugs}
{Alcohol consumption prohibited on site. Anyone clearly under the influence will be escorted off site.}
{Health and Safety Lead}
{2}
\hline

\end{longtable}
\end{landscape}

\begin{landscape}

\section{COVID-19}

The following COVID-19 risks have been considered for the Student Robotics Competition. 
This assessment is to be supplimented by the SUSU risk assessment.
Further description of the meaning of risk ratings (presented in this section as
$L \times S$) can be found in the final section.
All severities are deemed to be `3 day' injury/illness.
Current guidance advises isolating for 5 days, this is understood to be to prevent spreading rather than severity related.

\centering
\begin{longtable}{|p{17em}|p{8cm}|p{4cm}|p{4em}|}
\hline
\textbf{Hazard} & \textbf{Control Measures} & \textbf{Responsible Person} & \textbf{Risk Rating} \\
\hline
\endhead

\endfoot

\risk{Airbourne Transmission}
{
Anyone presenting symptoms or who has tested positive instructed not to attend.
Anyone who lives with someone who has tested positive instructed not to attend.
External windows and doors to be kept open.
All attendees asked to wear a mask outside of their pits and power tools area.
All attendees asked to respect each others personal space.
}
{Health and Safety Lead, Team Leaders}
{6}
\hline

\risk{Contact Transmission}
{
Anyone presenting symptoms or who has tested positive instructed not to attend.
Anyone who lives with someone who has tested positive instructed not to attend.
All attendees encouraged to wash hands regularly with soap or sanitiser.
Hand sanitiser available throughout the venue.
Common contact surfaces wiped down by SR Volunteers regularly during the event.
}
{Health and Safety Lead, SR Volunteers}
{6}
\hline

\risk{Event cancellation due to lack of volunteers}
{
Anyone presenting symptoms or who has tested positive instructed not to attend.
Anyone who lives with someone who has tested positive instructed not to attend.
All SR Volunteers to wear a mask during setup.
All attendees encouraged to wash hands regularly with soap or sanitiser.
All attendees asked to respect each others personal space.
Additional signage to be set up during the setup days to prevent others accessing the venue.
}
{Health and Safety Lead, SR Volunteers}
{1}
\hline

\end{longtable}

\end{landscape}


\input{assessment-guidance}

%\clearpage

%\newpage
%\includepdf[scale=1.0,landscape]{extras/Fire1.pdf}

\end{document}

