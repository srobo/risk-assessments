% Footnotes used inline in the text.
\newcommand{\manualhandlingfootnote}{\footnote{\url{http://www.hse.gov.uk/pubns/indg143.pdf}}}
\newcommand{\chargingfootnote}{\footnote{\url{https://studentrobotics.org/docs/kit/batteries/}}}
\newcommand{\estatesfacilitiesfootnote}{\footnote{Estates and Facilities: \url{http://www.southampton.ac.uk/estates/}}}
\newcommand{\hscfootnote}{\footnote{Health and Safety Coordinator; for the duration of this event, this role will be filled by Andrew Barrett-Sprot (\email{abarrett-sprot@studentrobotics.org}).}}


\newcommand{\groupname}{Student Robotics}
\newcommand{\assessorname}{Andrew Barrett-Sprot}
\newcommand{\assessoremail}{abarrett-sprot@studentrobotics.org}
\newcommand{\assessmentdate}{October 20, 2018}


\newcommand{\activityname}{Student Robotics 2019 - Kickstart}
\newcommand{\activitydate}{November 10, 2018}
\newcommand{\activitytime}{All day}
\newcommand{\activitylocation}{
    Thread, 1 Alie St, London E1 8DE, UK
}
\newcommand{\activitysummary}{
    This is the opening event to a five-month robotics competition in  which
    teams of students from schools will compete to design, build and test
    autonomous robots. The robots must perform the simple task of locating and
    moving marked boxes around an arena.

    The students are aged 16--18 and will be supervised by at least one teacher
    from their school at all times.

    The programme is run by the charity Student Robotics, who
    will be considered event staff at this and future events.
    
    The event has workshop sessions, which involves the teams working through a set of exercises
    to familiarise themselves with the robotics kit, with technical assistance
    from Student Robotics volunteers.
}


\newcommand{\references}{
    \reference{Guidance from the Health and Safety Executive, including manual
    handling procedures. \\
    \url{http://www.hse.gov.uk/risk/index.htm}}

    \reference{H\&S guidance from the Union Southampton website. \\
    \url{https://www.unionsouthampton.org/groups/admin/howto/protection}}

    \reference{Lithium polymer battery charging procedure, available on the
    Student Robotics website. \\
    \url{https://studentrobotics.org/docs/kit/batteries/}}

    \reference{Risk assessments prepared for previous events of a similar
    nature run by us.}
}


\newcommand{\risks}{
    \risk
        {Manual handling of heavy objects}
        {Competitors or staff could experience minor injury or back pains
         resulting from improper lifting methods.}
        {\item The HSE manual handling guidelines\manualhandlingfootnote are to
         be followed for all tasks involving heavy lifting.}
        {\item No further action required.}
        {1} % Likelihood (/3)
        {2} % Impact (/3)

    \risk
        {Obstacles on the floor, such as bags, boxes or trailing cables}
        {Competitors or staff may suffer injury as a result of tripping.}
        {\item Cables (such as laptop power supplies) will be routed underneath
         desks wherever possible.
         \item Cables that cannot be routed under desks will be clearly marked
         to increase their visibility.
         \item Bags, boxes and other items that are potential trip hazards will
         be stacked neatly by the walls whenever possible.}
        {\item No further action required.}
        {2} % Likelihood (/3)
        {1} % Impact (/3)

    \risk
        {Lithium polymer batteries}
        {LiPo batteries can ignite if damaged or misused, resulting in
         smoke/fire.}
        {\item Boxes containing batteries are clearly labelled as such and will
         be handled with care at all times.
         \item Batteries will be routinely inspected by staff for signs of
         damage or swelling, and set aside for safe disposal if necessary.
         \item Batteries are only to be charged in accordance with the recorded
         charging procedure\chargingfootnote.}
        {\item No further action required.}
        {1} % Likelihood (/3)
        {2} % Impact (/3)

    \risk
        {Interaction with autonomous robots/robotics equipment}
        {Competitors or staff could encounter minor injuries if the robots'
         actuators move unexpectedly.}
        {\item When robotics equipment is switched on, it will be treated as
         though its actuators could become active at any moment.}
        {\item The HSC\hscfootnote will verify that the robotics equipment does
         not present any sharp edges. If any are found, they will be removed,
         covered, or otherwise modified to reduce the chance and severity of
         injury they could cause.}
        {1} % Likelihood (/3)
        {1} % Impact (/3)

    \risk
        {Electrical equipment (robots, computers, battery chargers)}
        {Competitors or staff could get electrical shocks or burns from faulty
         equipment.}
        {\item Computing equipment, battery chargers and other mains-powered
         equipment has been PAT tested.
         \item Opened food and drink is prohibited in the vicinity of
         computers and robotics equipment.}
        {\item The HSC will verify that all wiring done by teams or already
         preexisting in the robotics kit is sufficiently insulated and robust
         before the robotics kit is allowed to be switched on.}
        {1} % Likelihood (/3)
        {2} % Impact (/3)

    \risk
        {Use of manual tools}
        {Competitors or staff could experience minor injury as a result of an
         accident or through improper use of tools.}
        {\item Care will be taken with tools to ensure that minimal injury
         results in the event of an accident.}
        {\item No further action required.}
        {2} % Likelihood (/3)
        {1} % Impact (/3)

    \risk
        {Allergies to fibres in Charging Bags}
        {Fibreglass charging bags are given for battery safety, which can
        cause skin irritation for people with sensitive skin or those with a
        fibreglass an allergy.}
        {\item Use gloves (any type) to minimise contact between fibres and hand, if persons
are known to be allergic to glass fibres.}
        {\item No further action required.}
        {1} % Likelihood (/3)
        {1} % Impact (/3)

    \risk
        {Burns from the boiling water tap}
        {A tap producing boiling water is in the kitchen area, it may cause burns if improperly used.}
        {\item Warn all attendees of the presence of the tap \& ensure it is obviously signposted.}
        {\item No further action required.}
        {1} % Likelihood (/3)
        {1} % Impact (/3)
}


\newcommand{\postrisks}{
    \subsection*{Risk of fire}

    To minimise the risk of fire resulting from this activity, food and drink
    will not be allowed near electrical equipment, and naked flames will be
    prohibited. The risk of fire occurring elsewhere in the building(s) is
    controlled primarily by the building operator.
    The HSC will ensure that all people present are informed of the
    locations of the exits and whether any fire drills are expected to take
    place. Should a fire break out (or any other event requiring evacuation),
    all people are to evacuate through the nearest accessible exit.
}


\input{template.tex}
